%        File: homework.tex
%     Created: 日 5月 06 01:00 下午 2018 C
% Last Change: 日 5月 06 01:00 下午 2018 C
%
\documentclass[UTF8,noindent]{ctexart}
\usepackage[a4paper,left=2.0cm,right=2.0cm,top=2.0cm,bottom=2.0cm]{geometry}
\usepackage{hyperref}
\usepackage{array}
\usepackage{url}
\usepackage{graphicx}
\usepackage{amsmath}
\usepackage{amssymb}
\usepackage{enumitem}
\usepackage{tikz}
\usepackage{float}
\usepackage{listings}
\usepackage{xcolor}
\usepackage{forest}
\usepackage{pgf}
\usepackage{multirow}
\usepackage{extarrows}
\usetikzlibrary{graphs}
\usetikzlibrary{arrows,automata}
\lstset{language = c,numbers=left, keywordstyle= \color{ blue!70 },commentstyle=\color{red!50!green!50!blue!50}, frame=shadowbox, rulesepcolor= \color{ red!20!green!20!blue!20 } 
} 
\usetikzlibrary{graphs}
\title{$Chapter\ 4-HW03$}
\author{$2015K8009929049$\ 冯吕}
\date{\today}
\begin{document}
\maketitle
\zihao{5}
\CJKfamily{zhsong}
$4.5.2$解:
\begin{itemize}
  \item $SS S+a*+$归约时使用的句柄为$SS+$;
	\item $SS+a*a+$规约时使用的句柄为$SS+$;
	  \item $aaa*a++$归约时使用的句柄为$a$;
\end{itemize}

$4.6.2$解:
增广文法为:
\begin{align*}
&0)\ S'\rightarrow S\\
  &1)\ S\rightarrow SS+\\
&2)\ S\rightarrow SS*\\
  &3)\ S\rightarrow a
\end{align*}

项目集和$GOTO$函数为:
\begin{align*}
  &I_0 : S'\rightarrow S, S\rightarrow .SS+, S\rightarrow .SS*, S\rightarrow .a \\
  & GOTO(I_0, S) = I_1, GOTO(I_0, a) = I_2;\\
  &I_1 :  S'\rightarrow S., S\rightarrow S.S+, S\rightarrow S.S*, S\rightarrow a., S\rightarrow .SS+, S\rightarrow .SS*, S\rightarrow .a \\
  &GOTO(I_1, S) = I_3, GOTO(I_1, a) = I_2, GOTO(I_2, \$) = acc;\\
  &I_2 :  S\rightarrow a. \\ 
  &I_3 : S\rightarrow SS.+, S\rightarrow SS.*, S\rightarrow S.S+, S\rightarrow S.S*, S\rightarrow .SS+, S\rightarrow .SS*, S\rightarrow .a\\
  & GOTO(I_3, S) = I_3, GOTO(I_3, a) = I_2, GOTO(I_3, +) = I_4, GOTO(I_3, *) = I_5;\\
  &I_4 :  S\rightarrow SS+.\\
  &I_5 :  S\rightarrow SS*.
\end{align*}

语法分析表为:
\begin{table}[htbp]
  \centering
  \begin{tabular}{|p{2cm}<{\centering}|p{2cm}<{\centering}|p{2cm}<{\centering}|p{2cm}<{\centering}|p{2cm}<{\centering}|p{2cm}<{\centering}|}
	\hline 
	\multirow{2}{*}{\text{状态}} & \multicolumn{4}{|c|}{$ACTION$} & $GOTO$\\
	\cline{2-6}
&	$a$ & $+$ & $*$ & $\$$ & $S$\\
	\hline
	$0$ & $S_2$ & & & & $1$\\
	\hline
	$1$ & $S_2$ & & & $acc$ & $3$\\
	\hline 
	$2$ & $R_3$ & $R_3$ & $R_3$ & $R_3$ & \\
	\hline
	$3$ & $S_2$ & $S_4$ & $S_5$ & & $3$ \\
	\hline
	$4$ & $R_1$ &$R_1$ & $R_1$ &$R_1$ & \\
	\hline
	$5$ & $R_2$ &$R_2$ & $R_2$ &$R_2$ & \\
	\hline
  \end{tabular}
\end{table}
%\begin{figure}[H]
%  \centering
%  \includegraphics[scale=0.4]{1.png}
%\end{figure}

该文法无冲突,是$SLR$文法。

$4.7.1$解:
增广文法为:
\begin{align*}
&0)S'\rightarrow S\\
&1)S\rightarrow SS+\\
&2)S\rightarrow SS*\\
&3)S\rightarrow a
\end{align*}

规范$LR$项目集族为:
\begin{align*}
  I_0: & [S'\rightarrow .S, \$], & [S\rightarrow .SS+, \$/a], & [S\rightarrow .SS*, \$/a], & [S\rightarrow .a, \$/a]\\
  I_1: & [S'\rightarrow S., \$],   & [S\rightarrow S.S+, \$/a], & [S\rightarrow S.S*, \$/a], & [S\rightarrow .SS+, a/+/*]\\
& [S\rightarrow .SS*, a/+/*] & [S\rightarrow .a, a/+/*] &\  &\ \\
  I_2:  & [ S\rightarrow a., a/\$ ] & & &\\
  I_3:  & [S\rightarrow SS.+, \$/a], & [S\rightarrow SS.*, \$/a], &[S\rightarrow S.S+, a/+/*], &[S\rightarrow S.S*, a/+/*]\\
       & [S\rightarrow .SS+, a/+/*], &[S\rightarrow .SS*, a/+/*], &[S\rightarrow .a, a/+/*] &\\
  I_4:  & [S\rightarrow SS.+, a/+/*], & [S\rightarrow SS.*, a/+/*], &[S\rightarrow S.S+, a/+/*], &[S\rightarrow S.S*, a/+/*]\\
       & [S\rightarrow .SS+, a/+/*], &[S\rightarrow .SS*, a/+/*], &[S\rightarrow .a, a/+/*] &\\
  I_5: & [S\rightarrow a., a/+/*]& & &\\
  I_6: &[S\rightarrow SS+., \$/a]& & &\\
  I_7: &[S\rightarrow SS*., \$/a] & & &\\
  I_8: &[S\rightarrow SS+., a/+/*]& & &\\
  I_9: &[S\rightarrow SS*., a/+/*]& & &
\end{align*}

语法分析表如下:
\begin{table}[htbp]
  \centering
  \begin{tabular}{|p{2cm}<{\centering}|p{2cm}<{\centering}|p{2cm}<{\centering}|p{2cm}<{\centering}|p{2cm}<{\centering}|p{2cm}<{\centering}|}
	\hline 
	\multirow{2}{*}{\text{状态}} & \multicolumn{4}{|c|}{$ACTION$} & $GOTO$\\
	\cline{2-6}
& $a$ & $+$ & $*$ & $\$$ & $S$\\
	\hline
	$0$ & $S_2$ & & & & $1$\\
	\hline
	$1$ & $S_5$ & & & $acc$ & $3$\\
	\hline 
	$2$ & $R_3$ &  & & $R_3$ & \\
	\hline
	$3$ & $S_5$ & $S_6$ & $S_7$ & & $4$ \\
	\hline
	$4$ & $S_5$ &$S_8$ & $S_9$ & & $4$ \\
	\hline
	$5$ & $R_3$ &$R_3$ & $R_3$ & & \\
	\hline
	$6$ & $S_1$ & & & $R_1$ & \\
	\hline
	$7$ & $R_2$ & & & $R_2$ & \\
	\hline
	$8$ & $R_1$ & $R_1$ & $R_1$ & & \\
	\hline
	$9$ & $R_2$ & $R_2$ & $R_2$ & & \\
	\hline
  \end{tabular}
\end{table}

%%\begin{figure}[H]
%  \centering
%  \includegraphics[scale=0.4]{2.png}
%\end{figure}

$LALR$项目集族为:
\begin{align*}
  I_0: & [S'\rightarrow .S, \$], & [S\rightarrow .SS+, \$/a], & [S\rightarrow .SS*, \$/a], & [S\rightarrow .a, \$/a]\\
  I_1: & [S'\rightarrow S., \$],   & [S\rightarrow S.S+, \$/a], & [S\rightarrow S.S*, \$/a], & [S\rightarrow .SS+, a/+/*]\\
& [S\rightarrow .SS*, a/+/*] & [S\rightarrow .a, a/+/*] &\  &\ \\
I_{25}:  & [ S\rightarrow a., a/+/*/\$ ] & & &\\
I_{34}:  & [S\rightarrow SS.+, \$/a/+/*], & [S\rightarrow SS.*, \$/a/+/*], &[S\rightarrow S.S+, a/+/*], &[S\rightarrow S.S*, a/+/*]\\
       & [S\rightarrow .SS+, a/+/*], &[S\rightarrow .SS*, a/+/*], &[S\rightarrow .a, a/+/*] &\\
  %I_4:  & [S\rightarrow SS.+, a/+/*], & [S\rightarrow SS.*, a/+/*], &[S\rightarrow S.S+, a/+/*], &[S\rightarrow S.S*, a/+/*]\\
       %& [S\rightarrow .SS+, a/+/*], &[S\rightarrow .SS*, a/+/*], &[S\rightarrow .a, a/+/*] &\\
  %I_5: & [S\rightarrow a., a/+/*]& & &\\
	   I_{68}: &[S\rightarrow SS+., \$/a/+/*]& & &\\
  I_{79}: &[S\rightarrow SS*., \$/a/+/*] & & &\\
  %I_8: &[S\rightarrow SS+., a/+/*]& & &\\
  %I_9: &[S\rightarrow SS*., a/+/*]& & &
\end{align*}

\end{document}


