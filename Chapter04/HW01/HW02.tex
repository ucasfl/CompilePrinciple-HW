%        File: HW02.tex
%     Created: 一 7月 09 09:00 下午 2018 C
% Last Change: 一 7月 09 09:00 下午 2018 C
%
\documentclass[UTF8,noindent]{ctexart}
\usepackage[a4paper,left=2.0cm,right=2.0cm,top=2.0cm,bottom=2.0cm]{geometry}
\usepackage{hyperref}
\usepackage{url}
\usepackage{graphicx}
\usepackage{amsmath}
\usepackage{amssymb}
\usepackage{enumitem}
\usepackage{tikz}
\usepackage{float}
\usepackage{listings}
\usepackage{xcolor}
\usepackage{forest}
\usepackage{pgf}
\usepackage{multirow}
\usepackage{extarrows}
\usetikzlibrary{graphs}
\usetikzlibrary{arrows,automata}
\lstset{language = c,numbers=left, showstringspaces = false, keywordstyle= \color{ blue!70 },commentstyle=\color{red!50!green!50!blue!50}, frame=shadowbox, rulesepcolor= \color{ red!20!green!20!blue!20 } 
} 
\usetikzlibrary{graphs}
\title{$Chapter\ 4-HW01$}
\author{$2015K8009929049$\ 冯吕}
\date{\today}
\begin{document}
\maketitle
\zihao{5}
\CJKfamily{zhsong}
$4.2.1$解:

$1)$串$aa+a*$的一个最左推导如下:
\[S\xLongrightarrow[lm]{} SS*\xLongrightarrow[lm]{} SS+S*\xLongrightarrow[lm]{} aS+S*\xLongrightarrow[lm]{} aa+S*\xLongrightarrow[lm]{} aa+a*\]

$2)$串$aa+a*$的一个最右推导如下:
\[S\xLongrightarrow[rm]{} SS*\xLongrightarrow[rm]{} Sa*\xLongrightarrow[rm]{} SS+a*\xLongrightarrow[rm]{} Sa+a*\xLongrightarrow[rm]{} aa+a*\]

$3)$该串的一棵语法分析树如下:
\begin{figure}[H]
\centering
\begin{forest}
  [{$S$}
	[{$S$}
	  [{$S$}
		[{$a$}]
	  ]
	  [{$S$}
		[{$a$}]
	  ]
	  [{$+$}]
	]
	[{$S$}
	  [{$a$}]
	]
	[{$*$}]
  ];
\end{forest}
\end{figure}

$4)$该文法不具有二义性。

$proof:$
\begin{itemize}
	\item 先证明一个该文法产生串的长度的结论:设串的推导过程中使用产生式$S\rightarrow S\ S\ +$和$S\rightarrow S\ S\ *$的次数为$m$,则串的长度为$L = 2\times m + 1$,且串中包含$m$个运算符和$m+1$个 a;
\begin{itemize}
  \item $1)$当$m=0$时,仅有$S\rightarrow a$一种情况,此时$L=1$,串由$1$个$a$和$0$个运算符构成,结论成立;
\item $2)$设当$m<k(k\ge 1)$时结论成立,则当$m=k$时,第一步推导必然为
\[S\rightarrow S_1S_2\ op\]
$op$为$+$或$*$。设$S_1\rightarrow \alpha, S_2\rightarrow \beta$, $\alpha, \beta$均为使用$S\rightarrow S_1S_2\ op$少于$k$次得到的串,设二者推导过程中分别使用该产生式$k_1$和$k_2$次,根据假设有:
\[L(\alpha) = 2*k_1 + 1, L(\beta) = 2*k_2+1\]
则串长度$L = L(\alpha) + L(\beta) + 1 = 2*(k_1+k_2+ 1) + 1= 2k+1$;且串中$a$的个数为$(k_1+1) + (k_2+1) = k+1$;运算符的个数为$k_1+k_2+1= k$,故结论成立。
\end{itemize}
\item 下面证明该文法无二义性,对串的长度做归纳。由前述证明可知,该文法产生的串长$L$可为任意非负奇数。对由该文法得到的长度为$K = 2*k+1$的串$w$:
  \begin{itemize}
	\item $1)$当$k=0$ 时,$L=1$,只有$S\rightarrow a$一种情况,显然没有二义性;
  \item $2)$设当$k<n$时结论成立。$S\rightarrow \omega$,根据$\omega$末尾运算符可确定第一步推导使用的产生式,不妨设为:
	\[S\rightarrow S_1S_2+\]
  从后向前处理串$\omega$,除去末尾的运算符,找到可以由$S$推导出的最短的串$\alpha$,设$\alpha$长度为$m_1$,由前述结论可知$m_1 = 2*k_1+1$,且$\alpha$包含$k_1$个运算符和$k_1+1$个$a$,由归纳假设可知$\alpha$无二义性,存在唯一的最左推导$S\rightarrow \alpha$;

  设串$\omega$剩余部分为$\beta$,设$\beta$的长度为$m_2$,同理可得$m_2 = 2*k_2+1$,$\beta$包含$k_2$个运算符与$k_2+1$个$a$,存在唯一最左推导$S\rightarrow \beta$,且满足$k = k_1+ k_2$。此时串$\omega$可表示成如下形式:
  \[\omega = \beta\alpha +\]

  故存在唯一的最左推导:
  \[S\rightarrow S\ S\ +\rightarrow \beta S\ + \rightarrow \beta \alpha +\]
  此时,仍不存在二义性。
  \end{itemize}
综上所述,该文法不具有二义性。
\end{itemize}

$5)$该文法生成的语言是运算数全为$a$,带有加法运算和乘法运算的后缀算术表达式全体。

$4.2.3\ a)$解:生成该语言的文法如下:
\[S\rightarrow 1S\mid 01S\mid \epsilon\]

$4.3.1$解:
$1)$该文法的每一个非终结符的多个选项之间,均没有非平凡的公共前缀,因此,该文法没有左公因子。

$2)$不能,因为该文法是左递归的($rexpr\rightarrow rexpr + rterm, rterm\rightarrow rterm \ rfactor$),所以,自顶向下分析会陷入无限循环。

$3)$消除左递归后的文法如下:
\begin{align*}
  &rexpr\rightarrow rterm\ rexpr' \\
  &rexpr' \rightarrow +rterm \ rexpr'\mid \epsilon\\
  &rterm \rightarrow rfactor \ rterm'\\
&rterm'\rightarrow rfactor \ rterm' \mid \epsilon\\
  &rfactor\rightarrow rprimary\ rfactor'\\
  &rfactor'\rightarrow *rfactor'\mid \epsilon\\
  &rprimary \rightarrow a\mid b
\end{align*}

$4)$消除左递归后,满足自顶向下分析的条件,因此适用于自顶向下的语法分析。
\end{document}


