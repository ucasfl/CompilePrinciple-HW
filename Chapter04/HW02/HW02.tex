%        File: HW02.tex
%     Created: 一 7月 09 09:00 下午 2018 C
% Last Change: 一 7月 09 09:00 下午 2018 C
%
\documentclass[UTF8,noindent]{ctexart}
\usepackage[a4paper,left=2.0cm,right=2.0cm,top=2.0cm,bottom=2.0cm]{geometry}
\usepackage{hyperref}
\usepackage{url}
\usepackage{graphicx}
\usepackage{amsmath}
\usepackage{amssymb}
\usepackage{enumitem}
\usepackage{tikz}
\usepackage{float}
\usepackage{listings}
\usepackage{xcolor}
\usepackage{forest}
\usepackage{pgf}
\usepackage{multirow}
\usepackage{extarrows}
\usetikzlibrary{graphs}
\usetikzlibrary{arrows,automata}
\lstset{language = c,numbers=left, showstringspaces = false, keywordstyle= \color{ blue!70 },commentstyle=\color{red!50!green!50!blue!50}, frame=shadowbox, rulesepcolor= \color{ red!20!green!20!blue!20 } 
} 
\usetikzlibrary{graphs}
\title{$Chapter\ 4-HW02$}
\author{$2015K8009929049$\ 冯吕}
\date{\today}
\begin{document}
\maketitle
\zihao{5}
\CJKfamily{zhsong}

$1.$解: 产生回溯的原因是,即使当非终结符用某个产生式匹配成功,但是这种成功可能只是暂时的,因为没
有足够的信息来唯一地确定可能的产生式,所以分析过程就会产生回溯。
不可以。例如对于产生式 $A\rightarrow \alpha\mid \beta$, $FIRST(\alpha)$与$FIRST(\beta)$交集为空集,但$\epsilon$是
其中某个$FIRST$ 集合的元素,不是一般性,假设$\epsilon\in FIRST(\alpha)$,想要避免回溯,则
还需要考虑$FOLLOW(A)$与$FIRST(\beta)$的情况。

$2.$解:$a$消除左递归后的文法如下:
\begin{align*}
  &lexp \rightarrow atom \mid list\\
  &atom \rightarrow \text{number}\mid \text{identifiler}\\
  &list \rightarrow (lexp-seq)\\
  &lexp-seq\rightarrow lexp\ A'\\
  &A'\rightarrow lexp\ A'\mid \epsilon
\end{align*}

$b.$ $FIRST$集合:
\begin{align*}
  &FIRST(lexp) = \{\text{number, identifiler}, (\}\\
	&FIRST(atom) = \{\text{number, identifiler}\}\\
	&FIRST(list) = \{\ (\ \}\\
	  &FIRST(\text{number}) = \{\text{number}\}\\
	&FIRST(\text{identifiler}) = \{\text{identifiler}\}\\
	  &FIRST(\ ( \ ) =  \{\ (\ \}\\
	&FIRST(\ ) \ ) = \{\ ) \ \}\\
	&FIRST(lexp-seq) = \{\text{number, identifiler}, ( \}\\
	  &FIRST(A') = \{\text{number, identifiler}, (, \epsilon\}\\
		&FIRST( \ (lexp-seq)\ ) = \{ ( \} \\
		  &FIRST(lexp \ A') = \{\text{number, identifiler}, (\}\\
			&FIRST(\ (lexp-seq\ ) = \{(\}\\
			  &FIRST( \ lexp-seq )\ ) = \{\text{number, identifiler}, (\}
\end{align*}

$FOLLOW$集合:
\begin{align*}
	%&FIRST(atom) = \{number, identifiler\}\\
	%&FIRST(list) = \{\ (\ \}\\
	  %&FIRST(number) = \{number\}\\
	  %&FIRST(identifiler) = \{identifiler\}\\
	  %&FIRST(\ ( \ ) =  \{\ (\ \}\\
	%&FIRST(\ ) \ ) = \{\ ) \ \}\\
  & FOLLOW(lexp) = \{\$, ) , number, identifiler, (\}\\
  & FOLLOW (atom)=\{ \$, ), number, identifier, ( \}\\ 
& FOLLOW (list)=\{ \$, ), number, identifier, ( \}\\
&FOLLOW(lexp-seq) = \{\ )\ \}\\
	  &FOLLOW(A') = \{\ )\ \}\\
	%&FOLLOW(\ lexp-se
\end{align*}

%$FOLLOW$集合:
$c.$ 证明:
\begin{itemize}
  \item 对于规则$lexp\rightarrow atom \mid list$,$atom$推出的串的首字符为number或identifiler,$list$推出的串的首字符为$($,满足条件;
	\item 对于规则$atom \rightarrow \text{number} \mid \text{identifiler}$,显然满足条件;
	\item 对于规则$A'\rightarrow lexp\ A'\mid \epsilon$,也满足$LL(1)$文法的条件;
\end{itemize}
因此,该文法是$LL(1)$文法。

$d.$ $LL(1)$分析表如下:

\begin{table}[htbp]
  \centering
  \begin{tabular}{|c|c|c|c|c|c|}
	\hline
	\multirow{2}{*}{\text{非终结符号}} &\multicolumn{5}{|c|}{\text{输入符号}}\\
  \cline{2-6}
  &number &identifiler & ( & ) & \$\\
  \hline
  $lexp$ &$lexp\rightarrow atom$ &$lexp\rightarrow atom$ &$lexp\rightarrow list$ &  & \\
  %\hline
  $atom$ &$atom \rightarrow\text{number}$ & $atom\rightarrow\text{identifiler}$ & &  &\\
  %\hline
  $list$ & & &$list\rightarrow(lexp-seq)$ &  &\\
  %\hline
  $lexp-seq$ &$lexp-seq\rightarrow lexp\ A'$ &$lexp-seq\rightarrow lexp \ A'$ &$lexp-seq\rightarrow lexp \ A'$ & & \\
  %\hline
 $A'$ &$A'\rightarrow lexp\ A'$ &$A'\rightarrow lexp\ A'$ &$A'\rightarrow lexp\ A'$ &$A'\rightarrow \epsilon$ &  \\
 \hline
\end{tabular}
\end{table}

$d.$ 对输入串$(a\ (b\ (2))\ (c))$,$LL(1)$分析程序的动作如下:
\begin{table}[htbp]
  \centering
  \begin{tabular}{r|r|l}
	\hline
	$STACK$ & $INPUT$ & $ACTION$\\
	\hline
	$lexp\$$ & $(a\ (b\ (2))\ (c))\$$ & \\
	$list\$$ & $(a\ (b\ (2))\ (c))\$$ & $output\ lexp \rightarrow list$\\
	$(lexp-seq)\$$ & $(a\ (b\ (2))\ (c))\$$ & $output\ list \rightarrow (lexp-seq)$\\
$lexp-seq)\$$ & $a\ (b\ (2))\ (c))\$$ & $match\ ($ \\
$lexp \ A')\$$ & $a\ (b\ (2))\ (c))\$$ & $output\ lexp-seq \rightarrow lexp \ A'$\\
$atom \ A')\$$ & $a\ (b\ (2))\ (c))\$$ & $output \ lexp\rightarrow atom$ \\
$aA')\$$ & $a\ (b\ (2))\ (c))\$$ & $output \ atom \rightarrow a$\\
$A')\$$ & $(b\ (2))\ (c))\$$ & $match\ a$\\
$lexp\ A')\$$ & $(b\ (2))\ (c))\$$ & $output \ A'\rightarrow lexp\ A'$ \\
$list \ A')\$$ &$(b\ (2))\ (c))\$$ &$output \ lexp\rightarrow list$\\
$(lexp-seq)A')\$$ & $(b\ (2))\ (c))\$$ & $output\ list\rightarrow (lexp-seq)$\\
$lexp-seq)A')\$$ & $b\ (2)) \ (c))\$$ & $match\ ($\\
$lexp\ A')A')\$$ & $b\ (2))\ (c))\$$ & $output\ lexp-seq \rightarrow lexp\ A'$\\
$atom \ A')A')\$$ & $b\ (2))\ (c))\$$ & $output\ lexp\rightarrow atom$\\
$b\ A')A')\$$ & $b\ (2))\ (c))\$$ & $output\ atom\rightarrow b$\\
$A')A')\$$ & $(2))\ (c))\$$ & $macth\ b$ \\
$lexp\ A')A')\$$ & $(2))\ (c))\$$ & $output\ A'\rightarrow lexp\ A'$\\
$list\ A')A')\$$ &$ (2))\ (c))\$$ & $output \ lexp\rightarrow list$\\
$(lexp-seq)  A')A')\$$ & $(2))\ (c))\$$ & $output\ list\rightarrow (lexp-seq)$\\
$lexp-seq) A')A')\$$ & $2)) \ (c))\$$ & $match\ ($\\
%$lexp-seq) A')A')$ & $2)) \ (c))$ & $match\ ($\\
$lexp\ A') A')A')\$$ & $2))\ (c))\$$ & $output\ lexp-seq \rightarrow lexp\ A'$\\
$atom \ A') A')A')\$$ & $2))\ (c))\$$ & $output\ lexp\rightarrow atom$\\
$2\ A') A')A')\$$ & $2))\ (c))\$$ & $output \ atom \rightarrow 2$\\
$A') A')A')\$$ & $))\ (c))\$$ & $match \ 2$\\
$)A')A')\$$ & $))\ (c))\$$ & $output\ A'\rightarrow \epsilon$\\
$A')A')\$$ & $)\ (c))\$$ &$match \ ($\\
$)A')\$$ & $)\ (c))\$$ & $output\ A'\rightarrow \epsilon$\\
$A')\$$ & $ (c))\$$ &$match \ ($\\
$lexp\ A')\$$ & $(c))\$$ & $output \ A'\rightarrow lexp\ A'$\\
$list\ A')\$$ & $(c))\$$ & $output\ lexp\ rightarrow list$\\
$(lexp-seq)A')\$$ & $(c))\$$ & $output \ list\rightarrow (lexp-seq)$\\
$lexp-seq)A')\$$ & $c))\$$ & $match \ ($\\
$lexp \ A')A')\$$ & $c))\$$ & $output lexp-seq\rightarrow lexp \ A'$ \\
$atom \ A')A')\$$ &$c))\$$ &$output lexp\rightarrow atom$\\
$c\ A')\ A')\$$ &$c))\$$ &$output\ atom \rightarrow c$\\
$A')A')$ & $))\$$ & $match \ c$\\
$)A')\$$ & $))\$$ & $output\ A'\rightarrow \epsilon$\\
$A')\$$ & $)\$$ &$match \ )$\\
$)\$$ & $)\$$ & $output\ A'\rightarrow \epsilon$\\
$\$$ & $\$$ &$match \ )$\\
  \end{tabular}
\end{table}

\end{document}


